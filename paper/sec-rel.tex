\section{Related Work \label{sec-rel}}
In this section, we will review the related studies in three aspects, namely recommender systems, heterogeneous information networks and network embedding.

In the literature of recommender systems, early works mainly adopt collaborative filtering (CF) methods to utilize historical interactions for recommendation~\cite{schafer2007collaborative}. Particularly, the matrix factorization approach~\cite{koren2009matrix,shi2012adaptive} has shown its effectiveness and efficiency in many applications, which factorizes user-item rating matrix into two low rank user-specific and item-specific matrices, and then utilizes the factorized matrices to make further predictions~\cite{koren2015advances}. Since CF methods usually suffer from cold-start problem, many works~\cite{yin2013lcars,feng2012incorporating,hong2013co} attempt to leverage additional information to improve recommendation performance. For example, Ma et al.~\cite{ma2011recommender} integrate social relations into matrix factorization in recommendation. Ling et al.~\cite{ling2014ratings} consider the information of both ratings and reviews and propose a unified model to combine content based filtering with collaborative filtering for rating prediction task. Ye et al.~\cite{ye2011exploiting} incorporate user preference, social influence and geographical influence in the recommendation and propose a unified POI recommendation framework. More recently, Sedhain et al.~\cite{sedhain2017low} explain drawbacks of three popular cold-start models~\cite{gantner2010learning,krohn2012multi,sedhain2014social} and further propose a learning based approach for the cold-start problem to leverage social data via randomised SVD. And many works begin to utilize deep models ($\eg$ convolutional neural network, auto encoder) to exploit text information~\cite{zheng2017joint}, image information~\cite{he2016vbpr} and network structure information~\cite{zhang2016collaborative} for better recommendation. In addition, there are also some typical frameworks focusing on incorporating auxiliary information for recommendation. Chen et al.~\cite{chen2012svdfeature} propose a typical SVDFeature framework to efficiently solve the feature based matrix factorization. And Rendle~\cite{rendle2010factorization} proposes factorization machine, which is a generic approach to combine the generality of feature engineering.

As a newly emerging direction, heterogeneous information network~\cite{shi2017survey} can naturally model complex objects and their rich relations in recommender systems, in which objects are of different types and links among objects represent different relations~\cite{sun2013mining,ou2013comparing}. And several path based similarity measures~\cite{lao2010relational,sun2011pathsim,shi2014hetesim} are proposed to evaluate the similarity of objects in heterogeneous information network. Therefore, some researchers have began to be aware of the importance of HIN based recommendation. Wang et al.~\cite{feng2012incorporating} propose the OptRank method to alleviate the cold-start problem by utilizing heterogeneous information contained in social tagging system. Furthermore, the concept of meta-path is introduced into hybrid recommender systems~\cite{yu2013recommendation}. Yu et al.~\cite{yu2013collaborative} utilize meta-path based similarities as regularization terms in the matrix factorization framework. Yu et al.~\cite{yu2014personalized} take advantage of different types of entity relationships in heterogeneous information network and propose a personalized recommendation framework for implicit feedback dataset. Luo et al.~\cite{luo2014hete} propose a collaborative filtering based social recommendation method using heterogeneous relations. More recently, Shi et al.~\cite{shi2015semantic} propose the concept of weighted heterogeneous information network and design a meta-path based collaborative filtering model to flexibly integrate heterogeneous information for personalized recommendation. In \cite{shi2016integrating,zheng2016dual,zheng2017recommendation}, the similarities of users and items are both evaluated by path based similarity measures under different semantic meta-paths and a matrix factorization based on dual regularization framework is proposed for rating prediction. Besides meta-path, Zhao et al.~\cite{zhao2017meta} propose a factorization machine based model integrated with meta-graph based similarity for recommendation. Most of HIN based methods rely on the path based similarity, which may not fully mine latent features of users and items on HINs for recommendation.


%As a newly emerging direction, HIN \cite{shi2017survey} can naturally model complex objects and their rich relations in recommender systems. \cite{yu2013collaborative} utilized the entity similarity extracted from HIN as a regularization term in CF.  \cite{shi2015semantic} employed the meta-path based similarity of users for personalized recommendation. Recently, \cite{zheng2017recommendation} utilized the similarities of users and items as the regularization term of matrix factorization. Most of HIN-based methods rely on the path based similarity, which cannot fully mine latent features of users and items on HINs for recommendation.

%On the other hand, network embedding has shown its potential in structure feature extraction and has been successfully applied in many data mining tasks~\cite{hoff2002latent,yan2007graph}, such as classification~\cite{tu2016max,kipf2016semi}, clustering~\cite{wei2017cross,cao2016deep} and recommendation~\cite{liang2016factorization,sunmrlr}. Deepwalk~\cite{perozzi2014deepwalk} combined random walk and skip-gram to learn network representations. Furthermore, Grover and Leskovec~\cite{grover2016node2vec} propose a more flexible network embedding framework based on a biased random walk procedure. In addition, LINE \cite{tang2015line} and SDNE~\cite{wang2016structural} characterize the second-order link proximity, as well as neighbor relations. Cao et al.~\cite{cao2015grarep} propose the GraRep model to capture higher-order graph proximity for network representations. Apart from leaning network embedding from only the topology, many works~\cite{pan2016tri,yang2015network,zhang2016homophily} begin to  leverage node content information and other available graph information for the robust representations. Pan et al.\cite{pan2016tri} propose the TriDNR model to learn optimal node representation with node structure information, node content and node labels. Yang et al.~\cite{yang2015network} propose the TADW model to incorporate text features of vertices into network representation based on matrix factorization framework. Zhang et al.~\cite{zhang2016homophily} consider neighbors homophily, topology structure and node content during network representation learning based on matrix decomposition. Most of network embedding methods focus on homogeneous networks, and thus they cannot directly be applied for heterogeneous networks. Although several works~\cite{chang2015heterogeneous,tang2015pte,xu2017embedding,chen2017task,dong2017metapath2vec} attempt to analyze heterogeneous networks via embedding methods, their representations of nodes and relations may not be suitable for recommendation.

On the other hand, network embedding has shown its potential in structure feature extraction and has been successfully applied in many data mining tasks~\cite{hoff2002latent,yan2007graph,cui2017survey,cui2018general}, such as classification~\cite{tu2016max}, clustering~\cite{wei2017cross,cao2016deep} and recommendation~\cite{liang2016factorization,sunmrlr}. Deepwalk~\cite{perozzi2014deepwalk} combines random walk and skip-gram to learn network representations. Furthermore, Grover and Leskovec~\cite{grover2016node2vec} propose a more flexible network embedding framework based on a biased random walk procedure. In addition, LINE \cite{tang2015line} and SDNE~\cite{wang2016structural} characterize the second-order link proximity, as well as neighbor relations. Cao et al.~\cite{cao2015grarep} propose the GraRep model to capture higher-order graph proximity for network representations. Besides leaning network embedding from only the topology,  there are also many works~\cite{pan2016tri,yang2015network,zhang2016homophily} leveraging node content information and other available graph information for the robust representations. Unfortunately, most of network embedding methods focus on homogeneous networks, and thus they cannot directly be applied for heterogeneous networks. Recently, several works~\cite{chang2015heterogeneous,tang2015pte,xu2017embedding,chen2017task,dong2017metapath2vec} attempt to analyze heterogeneous networks via embedding methods. Particularly, Chang et al.~\cite{chang2015heterogeneous} design a deep embedding model to capture the complex interaction between the heterogeneous data in the network. Xu et al.~\cite{xu2017embedding} propose a EOE method to encode the intra-network and inter-network edges for the coupled heterogeneous network. Dong et al.~\cite{dong2017metapath2vec} define the neighbor of nodes via meta-path and learn the heterogeneous embedding by skip-gram with negative sampling. More recently, Fu et al.~\cite{fu2017hin2vec} utilize a neural network model to capture rich relation semantics in HIN. Although these methods can learn network embeddings in various heterogeneous network, their representations of nodes and relations may not be optimum for recommendation.

%Pan et al.\cite{pan2016tri} propose the TriDNR model to learn optimal node representation with node structure information, node content and node labels. Yang et al.~\cite{yang2015network} propose the TADW model to incorporate text features of vertices into network representation based on matrix factorization framework. Zhang et al.~\cite{zhang2016homophily} consider neighbors homophily, topology structure and node content during network representation learning based on matrix decomposition. Most of network embedding methods focus on homogeneous networks, and thus they cannot directly be applied for heterogeneous networks. Although several works~\cite{chang2015heterogeneous,tang2015pte,xu2017embedding,chen2017task,dong2017metapath2vec} attempt to analyze heterogeneous networks via embedding methods, their representations of nodes and relations may not be suitable for recommendation.

To our knowledge, there are few attempts which adopt the network embedding approach to extract useful information from heterogeneous information network and leverage such information for rating prediction. The proposed approach utilizes the flexibility of HIN for modeling complex heterogeneous context information, and meanwhile borrows the capability of network embedding for learning effective information representation.
The final rating prediction component further incorporates a transformation mechanism implemented by three flexible functions to utilize the learned information from network embedding.
